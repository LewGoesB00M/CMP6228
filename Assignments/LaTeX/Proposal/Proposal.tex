% -------------------------------------------------------------------------------
% Establish page structure & font.
\documentclass[12pt]{report}

\usepackage[total={6.5in, 9in},
	left=1in,
	right=1in,
	top=1in,
	bottom=1in,]{geometry} % Page structure

\usepackage{graphicx} % Required for inserting images
\graphicspath{{../../.images}} % Any additional images I use (BCU logo, etc) are from here.

\usepackage[utf8]{inputenc} % UTF-8 encoding
\usepackage[T1]{fontenc} % T1 font
\usepackage{float}  % Allows for floats to be positioned using [H], which correctly
                    % positions them relative to their location within my LaTeX code.
\usepackage{subcaption}
\usepackage{csquotes}

% -------------------------------------------------------------------------------
% Declare biblatex with custom Harvard BCU styling for referencing.
\usepackage[
    useprefix=true,
    maxcitenames=3,
    maxbibnames=99,
    style=authoryear,
    dashed=false, 
    natbib=true,
    url=false,
    backend=biber
]{biblatex}

\usepackage[british]{babel}

% Additional styling options to ensure Harvard referencing format.
\renewbibmacro*{volume+number+eid}{
    \printfield{volume}
    \setunit*{\addnbspace}
    \printfield{number}
    \setunit{\addcomma\space}
    \printfield{eid}}
\DeclareFieldFormat[article]{number}{\mkbibparens{#1}}

\addbibresource{Proposal.bib}

% -------------------------------------------------------------------------------
% To prevent "Chapter N" display for each chapter
\usepackage[compact]{titlesec}
\usepackage{wasysym}
\usepackage{import}

\titlespacing*{\chapter}{0pt}{-2cm}{0.5cm}
\titleformat{\chapter}[display]
{\normalfont\bfseries}{}{0pt}{\Huge}

% -------------------------------------------------------------------------------
% Custom macro to make an un-numbered footnote.

\newcommand\blfootnote[1]{
    \begingroup
    \renewcommand\thefootnote{}\footnote{#1}
    \addtocounter{footnote}{-1}
    \endgroup
}

% -------------------------------------------------------------------------------
% Fancy headers; used to show my name, BCU logo and current chapter for the page.
\usepackage{fancyhdr}
\usepackage{calc}
\pagestyle{fancy}

\setlength\headheight{37pt} % Set custom header height to fit the image.

\renewcommand{\chaptermark}[1]{%
    \markboth{#1}{}} % Include chapter name.


% Lewis Higgins - ID 22133848           [BCU LOGO]                [CHAPTER NAME]
\lhead{Lewis Higgins - ID 22133848~~~~~~~~~~~~~~~\includegraphics[width=1.75cm]{BCU}}
\fancyhead[R]{\leftmark}

% ------------------------------------------------------------------------------
% Used to add PDF hyperlinks for figures and the contents page.

\usepackage{hyperref}

\hypersetup{
    colorlinks=true,
    linkcolor=black,
    filecolor=magenta,
    urlcolor=blue,
    citecolor=black,
}

% ------------------------------------------------------------------------------
\usepackage{xcolor} 
\usepackage{colortbl}
\usepackage{longtable}
\usepackage{amssymb}
% ------------------------------------------------------------------------------
\usepackage{tcolorbox}
\newcommand{\para}{\vspace{7pt}\noindent}
% -------------------------------------------------------------------------------

\title{Project Proposal}
\author{Lewis Higgins - Student ID 22133848}
\date{March 2025}

% -------------------------------------------------------------------------------

\begin{document}


\makeatletter
\begin{titlepage}
    \begin{center}
        \includegraphics[width=0.7\linewidth]{BCU}\\[4ex]
        {\huge \bfseries CMP6228 - Deep Learning Project}\\[2ex]
        {\large \bfseries  \@title}\\[50ex]
        {\@author}\\[2ex]
        {CMP6228 - Deep Neural Networks}\\[2ex]
        {Module Coordinator: Khalid Ismail}\\[2ex]
        {Word count excluding figures, references and appendices: XXXX / 1500}\\[10ex]
    \end{center}
\end{titlepage}
\makeatother
\thispagestyle{empty}
\newpage

% Page counter trick so that the contents page doesn't increment it.
\setcounter{page}{0}

\tableofcontents
\thispagestyle{empty}

\chapter*{Introduction}
\addcontentsline{toc}{chapter}{Introduction}
"This section should summarise and highlight the aim of
the proposal."


\chapter{Motivation and objectives}
"This section should include a description
of the selected dataset and a detailed description of the problem."

\section{Dataset choice}
\textbf{Going with the 
\href{https://data.mendeley.com/datasets/rscbjbr9sj/3}{PNEUMONIA dataset}} 
due to research backing and popularity in data science 
communities (Kaggle).


% ? Khalid said this dataset was okay, but you should still note that the datasets you've used in class 
% ? are measured in megabytes (MNIST is 11MB), whereas this is 1.2GB. Likely means the laptop can't do it. 
% ? The IMDB data of Lab 5 was the first to start stressing it, and that was only 17MB.

% ! The dataset images are not of consistent resolutions! Can that be addressed somehow?

% ? Show some sample data. Can just be the classes and how many examples of each class,
% ? you probably don't need to show an actual image from the set.
% ? You have two classes. Technically three, but identifying the third wouldn't be supervised learning,
% ? as viral/bacterial pneumonia is not labelled.

\section{Data science problem}

% "The aim is to develop a deep learning model to classify images into X classes."
% ? Describe any class imbalance and whatnot here.
% ! Massive class imbalance: Normal 1349, Pneumonia 3883.
% ? Describe the need for critical evaluation of the model.

% ! Footnote that the dataset contains multiple data science problems. I am only using "chest_xray", the 
% ! pneumonia identification problem.


\chapter{Related work}

% ! Added a ton of papers using this dataset to Zotero.
% ? You can use Litmaps to find more.

\section{Introduction}
"This section should demonstrate the main concepts
of related techniques that have been previously used to solve the problem."
\para What have other people done to solve it? How did they do it? 

% ! This is sectioned and divided exactly like your literature review.
% ! This section may be intensely difficult until the examples are uploaded for reference.

\section{Lit Review Topic 1}
% * Conventional machine learning methods (Random forest, SVM, etc)

\section{Lit Review Topic 2}


\section{Lit Review Topic 3}


% ?
% ! Example work has an extra chapter here titled "Concepts of Deep Learning" with 
% ! sections "Deep Neural Networks" and "Convolutional Neural Networks".
% ?

\chapter{Proposed model}
"This section should demonstrate the suitability of
the proposed solution in solving the data science problem"

% ? The model proposed here doesn't *need* to be the one you eventually use, but you should.


% ? When you submit your code, don't use your coloured comments like these.


\addcontentsline{toc}{chapter}{Bibliography}
\printbibliography

\end{document}