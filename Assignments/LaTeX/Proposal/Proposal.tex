% -------------------------------------------------------------------------------
% Establish page structure & font.
\documentclass[12pt]{report}

\usepackage[total={6.5in, 9in},
	left=1in,
	right=1in,
	top=1in,
	bottom=1in,]{geometry} % Page structure

\usepackage{graphicx} % Required for inserting images
\graphicspath{{../../.images}} % Any additional images I use (BCU logo, etc) are from here.

\usepackage[utf8]{inputenc} % UTF-8 encoding
\usepackage[T1]{fontenc} % T1 font
\usepackage{float}  % Allows for floats to be positioned using [H], which correctly
                    % positions them relative to their location within my LaTeX code.
\usepackage{subcaption}
\usepackage{csquotes}

% -------------------------------------------------------------------------------
% Declare biblatex with custom Harvard BCU styling for referencing.
\usepackage[
    useprefix=true,
    maxcitenames=3,
    maxbibnames=99,
    style=authoryear,
    dashed=false, 
    natbib=true,
    url=false,
    backend=biber
]{biblatex}

\usepackage[british]{babel}

% Additional styling options to ensure Harvard referencing format.
\renewbibmacro*{volume+number+eid}{
    \printfield{volume}
    \setunit*{\addnbspace}
    \printfield{number}
    \setunit{\addcomma\space}
    \printfield{eid}}
\DeclareFieldFormat[article]{number}{\mkbibparens{#1}}

\addbibresource{Proposal.bib}

% -------------------------------------------------------------------------------
% To prevent "Chapter N" display for each chapter
\usepackage[compact]{titlesec}
\usepackage{wasysym}
\usepackage{import}

\titlespacing*{\chapter}{0pt}{-2cm}{0.5cm}
\titleformat{\chapter}[display]
{\normalfont\bfseries}{}{0pt}{\Huge}

% -------------------------------------------------------------------------------
% Custom macro to make an un-numbered footnote.

\newcommand\blfootnote[1]{
    \begingroup
    \renewcommand\thefootnote{}\footnote{#1}
    \addtocounter{footnote}{-1}
    \endgroup
}

% -------------------------------------------------------------------------------
% Fancy headers; used to show my name, BCU logo and current chapter for the page.
\usepackage{fancyhdr}
\usepackage{calc}
\pagestyle{fancy}

\setlength\headheight{37pt} % Set custom header height to fit the image.

\renewcommand{\chaptermark}[1]{%
    \markboth{#1}{}} % Include chapter name.


% Lewis Higgins - ID 22133848           [BCU LOGO]                [CHAPTER NAME]
\lhead{Lewis Higgins - ID 22133848~~~~~~~~~~~~~~~\includegraphics[width=1.75cm]{BCU}}
\fancyhead[R]{\leftmark}

% ------------------------------------------------------------------------------
% Used to add PDF hyperlinks for figures and the contents page.

\usepackage{hyperref}

\hypersetup{
    colorlinks=true,
    linkcolor=black,
    filecolor=magenta,
    urlcolor=blue,
    citecolor=black,
}

% ------------------------------------------------------------------------------
\usepackage{xcolor} 
\usepackage{colortbl}
\usepackage{longtable}
\usepackage{amssymb}
% ------------------------------------------------------------------------------
\usepackage{tcolorbox}
\newcommand{\para}{\vspace{7pt}\noindent}
% -------------------------------------------------------------------------------

\title{Project Proposal}
\author{Lewis Higgins - Student ID 22133848}
\date{March 2025}

% -------------------------------------------------------------------------------

\begin{document}


\makeatletter
\begin{titlepage}
    \begin{center}
        \includegraphics[width=0.7\linewidth]{BCU}\\[4ex]
        {\huge \bfseries CMP6228 - Deep Learning Project}\\[2ex]
        {\large \bfseries  \@title}\\[50ex]
        {\@author}\\[2ex]
        {CMP6228 - Deep Neural Networks}\\[2ex]
        {Module Coordinator: Khalid Ismail}\\[2ex]
        {Word count excluding figures, references and appendices: XXXX / 1500}\\[10ex]
    \end{center}
\end{titlepage}
\makeatother
\thispagestyle{empty}
\newpage


% Page counter trick so that the contents page doesn't increment it.
\setcounter{page}{0}

\tableofcontents
\thispagestyle{empty}

\chapter*{Introduction}
\addcontentsline{toc}{chapter}{Introduction}
"This section should summarise and highlight the aim of
the proposal."

\para Khalid recommends that you use text or image data. Select a domain (medical, vehicles, etc)
and find a suitable dataset that can solve the problem you propose. Do any pneumonia datasets exist?
There's no named dataset size minimum. Consider a balanced size, as a large dataset of images will 
take huge amounts of time to train.

If there's some kind of diabetes-related dataset it might be possible to reuse CMP6202's extensive research.

\para \textbf{Datasets must be presented to Khalid to get his confirmation.} 


\chapter{Motivation and objectives}
"This section should include a description
of the selected dataset and a detailed description of the problem."

\chapter{Related work}
% Seems to be a literature review.
"This section should demonstrate the main concepts
of related techniques that have been previously used to solve the problem."

\para What have other people done to solve it? How did they do it? 

\chapter{Proposed model}
"This section should demonstrate the suitability of
the proposed solution in solving the data science problem"

\chapter*{Conclusion}
\addcontentsline{toc}{chapter}{Conclusion}
There doesn't seem to be a mandatory conclusion in the brief. Perhaps if one fits
it can be done, but it might not be necessary.


\addcontentsline{toc}{chapter}{Bibliography}
\printbibliography

\end{document}