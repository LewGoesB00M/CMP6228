% -------------------------------------------------------------------------------
% Establish page structure & font.
\documentclass[12pt]{report}

\usepackage[total={6.5in, 9in},
	left=1in,
	right=1in,
	top=1in,
	bottom=1in,]{geometry} % Page structure

\usepackage{graphicx} % Required for inserting images
\graphicspath{{../.images/}} % Any additional images I use (BCU logo, etc) are from here.

\usepackage[utf8]{inputenc} % UTF-8 encoding
\usepackage[T1]{fontenc} % T1 font
\usepackage{float}  % Allows for floats to be positioned using [H], which correctly
                    % positions them relative to their location within my LaTeX code.
\usepackage{subcaption}
\usepackage{csquotes}

% -------------------------------------------------------------------------------
% Declare biblatex with custom Harvard BCU styling for referencing.
\usepackage[
    useprefix=true,
    maxcitenames=3,
    maxbibnames=99,
    style=authoryear,
    dashed=false, 
    natbib=true,
    url=false,
    backend=biber
]{biblatex}

\usepackage[british]{babel}

% Additional styling options to ensure Harvard referencing format.
\renewbibmacro*{volume+number+eid}{
    \printfield{volume}
    \setunit*{\addnbspace}
    \printfield{number}
    \setunit{\addcomma\space}
    \printfield{eid}}
\DeclareFieldFormat[article]{number}{\mkbibparens{#1}}

\addbibresource{Report.bib}

% -------------------------------------------------------------------------------
% To prevent "Chapter N" display for each chapter
\usepackage[compact]{titlesec}
\usepackage{wasysym}
\usepackage{import}

\titlespacing*{\chapter}{0pt}{-2cm}{0.5cm}
\titleformat{\chapter}[display]
{\normalfont\bfseries}{}{0pt}{\Huge}

% -------------------------------------------------------------------------------
% Custom macro to make an un-numbered footnote.

\newcommand\blfootnote[1]{
    \begingroup
    \renewcommand\thefootnote{}\footnote{#1}
    \addtocounter{footnote}{-1}
    \endgroup
}

% -------------------------------------------------------------------------------
% Fancy headers; used to show my name, BCU logo and current chapter for the page.
\usepackage{fancyhdr}
\usepackage{calc}
\pagestyle{fancy}

\setlength\headheight{37pt} % Set custom header height to fit the image.

\renewcommand{\chaptermark}[1]{%
    \markboth{#1}{}} % Include chapter name.


% Lewis Higgins - ID 22133848           [BCU LOGO]                [CHAPTER NAME]
\lhead{Lewis Higgins - ID 22133848~~~~~~~~~~~~~~~\includegraphics[width=1.75cm]{BCU}}
\fancyhead[R]{\leftmark}

% ------------------------------------------------------------------------------
% Used to add PDF hyperlinks for figures and the contents page.

\usepackage{hyperref}

\hypersetup{
    colorlinks=true,
    linkcolor=black,
    filecolor=magenta,
    urlcolor=blue,
    citecolor=black,
}

% ------------------------------------------------------------------------------
\usepackage{xcolor} 
\usepackage{colortbl}
\usepackage{longtable}
\usepackage{amssymb}
% ------------------------------------------------------------------------------
\usepackage{tcolorbox}
\newcommand{\para}{\vspace{7pt}\noindent}
% -------------------------------------------------------------------------------

\title{Untitled CMP6228 Report}
\author{Lewis Higgins - Student ID 22133848}
\date{Unknown, 2025}

% -------------------------------------------------------------------------------

\begin{document}


\makeatletter
\begin{titlepage}
    \begin{center}
        \includegraphics[width=0.7\linewidth]{BCU}\\[4ex]
        {\huge \bfseries CMP6228 - Unknown Assignment}\\[2ex]
        {\large \bfseries  \@title}\\[50ex]
        {\@author}\\[2ex]
        {CMP6228 - Deep Neural Networks}\\[2ex]
        {Module Coordinator: Khalid Ismail}\\[10ex]
    \end{center}
\end{titlepage}
\makeatother
\thispagestyle{empty}
\newpage


% Page counter trick so that the contents page doesn't increment it.
\setcounter{page}{0}

\tableofcontents
\thispagestyle{empty}

\chapter*{Introduction}
\addcontentsline{toc}{chapter}{Introduction}
This module consists of two assignments in a similar fashion to CMP6230. 
By week 8, you must have found and written a \textbf{proposal} (20\%) detailing 
the problem to be solved with deep learning.

\para The second assignment is worth 80\%, and is a full report detailing the design, development
and evaluation of a deep learning model to solve the proposed problem. From this, an immediate 
observation is once again the medical field - image classification of people with some kind of disease 
from an X-ray dataset of some sort. Recall previous data sources - Kaggle, IEEE DataPort, UCI ML. Try 
to avoid the same topics as others if possible, though that's perhaps even less likely than it was in other 
modules if image classification is the main objective.

\para \textbf{Assignment 2's deadline is presumably in Week 12, or maybe even later. 
\textit{A draft version is due in Week 11, however.} } 

\para It is unlikely this module can be completed on your laptop, as TensorFlow 
will be used. Colab is proposed as an option, which could perhaps be feasible enough, though 
that may likely depend on the size of the chosen dataset, of which no specific value 
count was given. (Nor feature count, though this is likely irrelevant in an image classification scenario).
Colab ran epochs \~30-40\% quicker at first glance if the runtime was changed to a GPU.
You should fine-tune a good amount of epochs; more epochs just increase training time, but will they actually 
enhance the model that much or will it just plateau? A 3060 Ti processes twice as fast as Colab's free GPU.

\para Because of this assignment being almost identical to CMP6202 (only difference is the model and 
the fact that you'll likely be doing image classification rather than text),
it's likely that elements of it can be adapted to Assignments 1 and 2. It's likely to be a very math-heavy 
module.

\para The Moodle page for this module is not immediately available; as such, you can't really plan ahead 
for what will be done on a week-by-week basis. Because of this uncertainty, it might be best to prioritise 
this module over CMP6207.

\chapter*{Tech stack}
\begin{itemize}
    \item Previously seen libraries:
    \begin{itemize}
        \item Conda 
        \item Numpy
        \item Scipy (sort-of seen before but not properly)
        \item Scikit-learn 
    \end{itemize}
    \item Pillow
    \begin{itemize}
        \item An image processing library.
    \end{itemize}
    \item h5py
    \begin{itemize}
        \item Appears to be used for serialization of models, like Pickle was in CMP6230.
    \end{itemize}
    \item Keras \& TensorFlow
    \begin{itemize}
        \item Keras apparently uses TensorFlow to maximise efficiency with tensor calculations.
        \item (Or the other way around. Keras is a part of TensorFlow now.)
        \item A tensor is a multidimensional array or matrix.
    \end{itemize}
\end{itemize}

\para Keras has two ways of composing models: Sequential and Functional. \textbf{Only sequential will be 
used in the scope of this module.} 
\vspace{8pt}
\subsubsection{Sequential}
Multiple predefined models are stacked in a linear pipeline like a stack or queue:

% The second line of this seems to input a NumPy array, as there's no second axis.
\begin{verbatim}
    model = Sequential()
    model.add(Dense(N_HIDDEN, input_shape=(784,)))
    model.add(Activation('relu'))
    model.add(Dropout(DROPOUT))
    model.add(Dense(N_HIDDEN))
    model.add(Activation('relu'))
    model.add(Dropout(DROPOUT))
    model.add(Dense(nb_classes))
    model.add(Activation('softmax'))
    model.summary()
\end{verbatim}

\subsubsection{Functional}
Uses an API where complex models can be defined such as DAGs (!!!), models with shared layers, 
or multi-output models. Not much info was given on this, likely requiring your own research.
\textbf{Although it's good to know that these exist, they won't be used in this module.}  


\chapter*{Conclusion}
\addcontentsline{toc}{chapter}{Conclusion}

Overall, something was done\dots

\begingroup
\renewcommand\thechapter{A}
\titleformat{\chapter}[display]
{\normalfont\huge\bfseries}{}{20pt}{\Huge}
\setcounter{section}{0} 
\setcounter{figure}{0}

\chapter*{Appendix A - Model Training}
\addcontentsline{toc}{chapter}{Appendix A - Model Training}
\markboth{Appendix A}{}

Training convolutional neural networks is a highly intensive task, requiring substantial amounts of processing power,
meaning many have to turn to cloud service providers to train their models.
Many such options exist to find the necessary processing power to train an advanced CNN, such as Google Colab, which offers 
free (rate-limited) access to Tesla T4 GPUs. Personal experimentation has revealed that it can be difficult to leverage Colab 
when performing data augmentation, often encountering rate limitations resulting in forced session termination. Additionally,
model training can take a long time on these GPUs, and there is always the possibility of a sudden internet connection loss that could 
cause all work to be lost.

\para Therefore, which Colab likely could be suitable for training the proposed model, it is instead also 
possible to leverage a local GPU which will be able to train the model at a substantially faster rate.
The proposed training method will be a local Jupyter Notebook running on a desktop PC with an AMD Radeon RX 9070 XT graphics card.
This GPU is not only substantially faster than a Tesla T4 as per \textcite{hardwaredbTeslaT4Vs,technicalcityRX9070XT} as well as personal 
anecdotal experimentation seen in Figures \ref{fig:GPUComparisonA} and \ref{fig:GPUComparisonB},
but would also not impose rate limits due to it being personally owned. This should allow for the model to be trained much faster, while not 
having to depend on cloud service availability.

\begin{figure}[H]
    \centering
    \includegraphics[width=0.5\textwidth]{Proposal/ColabModelTraining.png}
    \caption{Training time for a CNN on the CIFAR10 dataset using Google Colab.\label{fig:GPUComparisonA}}
\end{figure}

\begin{figure}[H]
    \centering
    \includegraphics[width=0.5\textwidth]{Proposal/9070XTModelTraining.png}
    \caption{Training time for a CNN on the CIFAR10 dataset using an RX 9070 XT.\label{fig:GPUComparisonB}}
\end{figure}

\endgroup 

\addcontentsline{toc}{chapter}{Bibliography}
\printbibliography

\end{document}